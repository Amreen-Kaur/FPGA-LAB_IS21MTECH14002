\documentclass{article}
\usepackage[utf8]{inputenc}
\usepackage{amsmath}
\usepackage[utf8]{inputenc}
\usepackage[T1]{fontenc}
\usepackage{karnaugh-map}
\usetikzlibrary{calc}
\usepackage{placeins}
\usepackage{amsfonts}
\usepackage{multicol}
\usepackage{amssymb}
\usepackage[framed]{matlab-prettifier}
\usepackage{graphicx}
\usepackage[margin=0.75in]{geometry}
\usepackage{enumerate}
\usepackage{circuitikz}

\title{Assignment-1}
\author{Amreen Kaur (IS21MTECH14002)}
\date{15 January 2022}

\begin{document}

\maketitle

\section{Question}
Derive a Canonical POS expression for a Boolean function FN,
represented by the following truth table : 

\begin{center}
\begin{tabular}{ |c|c|c|c| } 
 \hline
X & Y & Z & FN(X,Y,Z) \\ 
0 & 0 & 0 & 1\\ 
0 & 0 & 1 & 1\\ 
0 & 1 & 0 & 0\\ 
0 & 1 & 1 & 0\\ 
1 & 0 & 0 & 1\\ 
1 & 0 & 1 & 0\\ 
1 & 1 & 0 & 0\\ 
1 & 1 & 1 & 1\\ 

 \hline
\end{tabular}
\end{center}

\section{Answer}
The  Canonical POS expression for the Boolean function FN can be represented as:$\pi$ M(2,3,5,6) and the expression is as follows:-

\\The Boolean Function is: 

\boxed{(X+Y'+Z).(X+Y'+Z').(X'+Y+Z').(X'+Y'+Z)}

\\Now, the reduced or non-canonical form of the expression is as follows:-

\\The Boolean Function is: 

\boxed{(X+Y')(Y'+Z)(X'+Y+Z')}

\\The reduced expression was obtained using the following k-map.



\begin{figure}[!ht]
\centering
\resizebox{\columnwidth}{!}
{
\begin{karnaugh-map}[4][2][1][][]
    \minterms{0,1,4,7}
    \maxterms{2,3,5,6}
    \implicant{3}{2}
    \implicant{2}{6}
    \implicant{5}{5}
    \draw[color=blue, ultra thin] (0,2) --
    node [pos=0.7, above right, anchor=south west] {$YZ$}
    node [pos=0.7, below left, anchor=north east] {$X$} 
    ++(130:1);
    \end{karnaugh-map}
}
\end{figure}
\end{document}
